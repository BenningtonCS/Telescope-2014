% !TEX TS-program = pdflatex
% !TEX encoding = UTF-8 Unicode

% This is a simple template for a LaTeX document using the "article" class.
% See "book", "report", "letter" for other types of document.

\documentclass[11pt]{article} % use larger type; default would be 10pt

\usepackage[utf8]{inputenc} % set input encoding (not needed with XeLaTeX)

%%% Examples of Article customizations
% These packages are optional, depending whether you want the features they provide.
% See the LaTeX Companion or other references for full information.

%%% PAGE DIMENSIONS
\usepackage{geometry} % to change the page dimensions
\geometry{a4paper} % or letterpaper (US) or a5paper or....
% \geometry{margin=2in} % for example, change the margins to 2 inches all round
% \geometry{landscape} % set up the page for landscape
%   read geometry.pdf for detailed page layout information

\usepackage{graphicx} % support the \includegraphics command and options

% \usepackage[parfill]{parskip} % Activate to begin paragraphs with an empty line rather than an indent

%%% PACKAGES
\usepackage{booktabs} % for much better looking tables
\usepackage{array} % for better arrays (eg matrices) in maths
\usepackage{paralist} % very flexible & customisable lists (eg. enumerate/itemize, etc.)
\usepackage{verbatim} % adds environment for commenting out blocks of text & for better verbatim
\usepackage{subfig} % make it possible to include more than one captioned figure/table in a single float
% These packages are all incorporated in the memoir class to one degree or another...

%%% HEADERS & FOOTERS
\usepackage{fancyhdr} % This should be set AFTER setting up the page geometry
\pagestyle{fancy} % options: empty , plain , fancy
\renewcommand{\headrulewidth}{0pt} % customise the layout...
\lhead{}\chead{}\rhead{}
\lfoot{}\cfoot{\thepage}\rfoot{}

%%% SECTION TITLE APPEARANCE
\usepackage{sectsty}
\allsectionsfont{\sffamily\mdseries\upshape} % (See the fntguide.pdf for font help)
% (This matches ConTeXt defaults)

%%% ToC (table of contents) APPEARANCE
\usepackage[nottoc,notlof,notlot]{tocbibind} % Put the bibliography in the ToC
\usepackage[titles,subfigure]{tocloft} % Alter the style of the Table of Contents
\renewcommand{\cftsecfont}{\rmfamily\mdseries\upshape}
\renewcommand{\cftsecpagefont}{\rmfamily\mdseries\upshape} % No bold!

%%% END Article customizations

%%% The "real" document content comes below...

\title{\huge{Bennington College Small Radio Telescope} \\ Operations Manual}
\author{}
\date{} % Activate to display a given date or no date (if empty),
         % otherwise the current date is printed 

\begin{document}
\maketitle


\vspace{4cm}

\begin{center}
\emph{\Large{Andrew Cencini – Hugh Crowl} \\ 
\large{William Buchanan, Alexander Curth, Erick Daniszewski, Evan Gall,
Clemente Gilbert-Espada, Chernoh Jalloh, Brendon Walter}}
\end{center}


\vspace{7cm}

\begin{center}
\emph{with guidance from} \\ 
\Large{\textbf{MIT Haystack Observatory's}} \\ 
Haystack Small Radio Telescope Project
\end{center}
\normalsize

\tableofcontents


\section{Introduction}

This manual serves as a guide following the work done at Bennington College during the spring of 2014 to build and operate a radiotelescope, following the procedure layed out by MIT Haystack and their Small Radio Telescope (SRT) project. This manual contains information regarding both the hardware and software components of the project which should allow for a simple and successful build and operation of a SRT.

\section{Parts List}

\section{Tools List}

\section{Installing SRTN Software}

\subsection{Ubuntu}

\subsection{CentOS}

\subsection{Arch}

\section{RAS SPID Rotator}

\subsection{Setup}

\subsection{Wiring}

\subsection{Controller Settings}

\section{Building Components}

\subsection{LNA}

\subsection{Feed}

\subsection{Roof Mount}

\subsection{Dish}

\end{document}
